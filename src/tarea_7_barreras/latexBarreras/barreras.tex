%! Author = Daryl Butron
%! Date = 27/04/2021

% Preamble
\documentclass[11pt]{article}

% Packages
\usepackage[utf8]{inputenc}
\usepackage[table]{xcolor}
\usepackage[spanish]{babel}
%\usepackage{hyperref}

\usepackage{amsmath}
\usepackage{tabularx}
\usepackage{natbib}
\usepackage{listings}
\usepackage{graphicx}
%\usepackage[T1]{fontenc}


\title{Barreras en memoria compartida}
\author{daryl.butron@ucsp.edu.pe}
\date{Abril 2021}

% Defines
\definecolor{codegreen}{rgb}{0,0.6,0}
\definecolor{codegray}{rgb}{0.5,0.5,0.5}
\definecolor{codepurple}{rgb}{0.58,0,0.82}
\definecolor{backcolour}{rgb}{0.95,0.95,0.92}
\definecolor{applegreen}{rgb}{0.55, 0.71, 0.0}
\definecolor{ao(english)}{rgb}{0.0, 0.5, 0.0}

\lstdefinestyle{mystyle}{
    backgroundcolor=\color{backcolour},
    commentstyle=\color{codegreen},
    keywordstyle=\color{ao(english)},
    numberstyle=\tiny\color{codegray},
    stringstyle=\color{codepurple},
    basicstyle=\ttfamily\footnotesize,
    breakatwhitespace=false,
    breaklines=true,
    captionpos=b,
    keepspaces=true,
    numbers=left,
    numbersep=5pt,
    showspaces=false,
    showstringspaces=false,
    showtabs=false,
    tabsize=2
}

% Document
\begin{document}
    \maketitle

    \section{Introducción}
    Hay un problema en la programación con memoria compartida,
    el problema es sobre sincronizar los subprocesos para asegurarnos
    que todos estén en un punto sobre algun programa dado.
    Esta sincronización es llamada barrera, ya que ningún thread
    puede seguir avanzando mas allá de esta barrera dada. Esto tiene
    varias aplicaciones como por ejemplo revisar si una vez terminada
    la carrera de los threads verificar cual fue mas lento al finalizar
    una tarea dada a este thread.


    \section{Busy-Wait y Mutex}
    Para implementar una barrera usando Busy-Wait y un mutex
    se requiere hacer lo siguiente:
    \begin{enumerate}
        \item Usar un contador compartido que este protegido por el mutex.
        \item Verificar con el contador si todos los subprocesos ingresaron a sección crítica
        \item Si se cumple el paso anterior, entonces dejar un bucle busy-wait.
    \end{enumerate}
    Esta estrategia desperdiciara ciclos en CPU cuando los
    threads esten bucle busy-wait, además si ejecutamos un programa
    en el cual le damos mas threads que nucleos que pueda soportar
    entonces se producira degradación de perfomance al código.

%%%%%%%%%%%%%%%%%%%%%%%%%%%%%%%%%%%%%%%%%%%%
    \section{Semaforo}
    Aqui también se utiliza un contador para saber
    cuantos threads han ingresado a la barrera además de otro
    contador que sera para contar los semaforos.
    Aqui no importa si el thread esta ejecutando el bucle
    de llamadas o si estan bloqueadas, ya que eventualmente
    estas terminaran y llegaran al mismo punto.
    Esta estrategia de usar semaforos es mejor que la anterior,
    ya que los threads no consumen ciclos en CPU cuando estan
    bloqueadas en un semaforo de espera.

    %%%%%%%%%%%%%%%%%%%%%%%%%%%%%%%%%%%%%
    \section{Variable Condicional}
    Este utliza un objeto llamado variable de condición que permite a los
    threads se encuentren en modo de suspensión de ejecución
    hasta que suceda algo que active la variable de condición
    y por esto despierte. Una variable de condición siempre esta
    asociado con un mutex.
    Para trabajar con esta estrategia se requiere el tipo de dato
    pthread_cond_t para hacer de señales o broadcast.
    Es esencial que se llame correctamente pthread_cond_wait
    para desbloquear los mutex, ya que sino es podría haber un
    cuello de botella cuando llegue por carrera el primer thread
    y este bloquee la variable de condicion de espera.

    \section{Barrera de Pthreads}
    Para este caso se requiere del uso de 3 metodos:
    \begin{enumerate}
        \item pthread_barrier_init para inicializar la funcion y los threads
            requeridos a usar
        \item pthread_barrier_wait para hacer la espera de carrera
        \item pthread_barrier_destroy para destruir y unir los threads
    \end{enumerate}


    \section{Observaciones y Conclusiones}
    \begin{enumerate}
        \item La estrategia de usar Busy-Wait junto con el mutex
        es la peor solución posible en caso de usar clocks en CPU
        \item Las variables de condición trabajan como un join
            en threads
        \item Propiamente en pthreads se ha desarrollado 3 funcionalidades
        que permiten inicializar, esperar y cerrar o destruir los threads
        una vez estos terminen sus procesos en carrera.
    \end{enumerate}

\end{document}