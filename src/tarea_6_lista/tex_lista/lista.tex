%! Author = Daryl Butron
%! Date = 27/04/2021

% Preamble
\documentclass[11pt]{article}

% Packages
\usepackage[utf8]{inputenc}
\usepackage[table]{xcolor}
\usepackage[spanish]{babel}
%\usepackage{hyperref}

\usepackage{amsmath}
\usepackage{tabularx}
\usepackage{natbib}
\usepackage{listings}
\usepackage{graphicx}
%\usepackage[T1]{fontenc}

\title{Lista Enlazada Paralelo con Pthreads}
\author{daryl.butron@ucsp.edu.pe}
\date{}

% Defines
\definecolor{codegreen}{rgb}{0,0.6,0}
\definecolor{codegray}{rgb}{0.5,0.5,0.5}
\definecolor{codepurple}{rgb}{0.58,0,0.82}
\definecolor{backcolour}{rgb}{0.95,0.95,0.92}
\definecolor{applegreen}{rgb}{0.55, 0.71, 0.0}
\definecolor{ao(english)}{rgb}{0.0, 0.5, 0.0}

\lstdefinestyle{mystyle}{
    backgroundcolor=\color{backcolour},
    commentstyle=\color{codegreen},
    keywordstyle=\color{ao(english)},
    numberstyle=\tiny\color{codegray},
    stringstyle=\color{codepurple},
    basicstyle=\ttfamily\footnotesize,
    breakatwhitespace=false,
    breaklines=true,
    captionpos=b,
    keepspaces=true,
    numbers=left,
    numbersep=5pt,
    showspaces=false,
    showstringspaces=false,
    showtabs=false,
    tabsize=2
}

% Document
\begin{document}

    \maketitle

    \section{Link al repositorio}
    https://github.com/dabc312GitHub/tarea678.git

    \section{Funciones}
    Se tienen principalmente 3 funciones las cuales crearan problemas en
    tratar una estructura que almacene elementos:
    \begin{enumerate}
        \item Member: Esta se encargará solo de recorrer la lista hasta encontrar el
              elemento o sea nulo.
        \item Insert: Esta se encargará de crear nodos y agregar el valor correspondiente
              en el determinado nodo de forma que sea una lista ordenada.
        \item Delete: Esta se encargará de eliminar nodos con su respectivo valor.
    \end{enumerate}

    \section{Contexto Paralelo}
    Para trabajar con paralelismo con pthreads se requiere saber como solucionar el tema
    de colisiones entre threads al realizar el uso de un recurso compartido en simultáneo.
    Para esto se tendra que gestionar adecuadamente un recurso compartido como el puntero
    de tipo nodo head_p como variable global, por lo cual se presentarán los siguientes casos
    por solución presentada:

    \subsection[]{Solución 1: }
    Se podría colocar un bloqueo por mutex antes de la operación y colocar el desbloqueo
    despues de la operación. Tratandose así, la lista se bloquea y desbloquea por operación.
    \begin{lstlisting}[
        language = C++,
        caption = Solución 1
    ]
    Pthread_mutex_lock(&list mutex);
    Member(value);
    Pthread_mutex_unlock(&list mutex);
    \end{lstlisting}
    Problema: Esta solución covierte a las operaciones de forma secuencial

    \subsection{Solución 2:}
    Se podrían bloquear nodos en lugar de toda la lista, para que así se puede tener
    mayor libertad de manejo de las operaciones en simultáneo. Para que esta solución
    funcione se procede a crear un mutex por nodo, es decir que en la misma estructura
    del nodo se tiene el mutex.
    \begin{lstlisting}[
        language = C++,
        caption = Solución 2 en la operación Member
    ]

    struct list_node_s {
      int data;
      struct list_node_s* next;
      pthread_mutex_t mutex;
    };

    struct list_node_s** head_p;
    struct pthread_mutex_t ** head_p_mutex;

    void MemberThreads(int value) {
      struct list_node_s* temp_p;
      pthread_mutex_lock((pthread_mutex_t *) &head_p_mutex);
      temp_p = (struct list_node_s *) head_p;
      while (temp_p != NULL && temp_p->data < value) {
        if (temp_p->next != NULL)
          pthread_mutex_lock(&temp_p->next->mutex);
        if (temp_p == *head_p)
          pthread_mutex_unlock((pthread_mutex_t *) &head_p_mutex);
        pthread_mutex_unlock(&temp_p->mutex);
        temp_p = temp_p->next;
      }
    }
    /* MemberPthreads */
    \end{lstlisting}
    Problema: Al agregar el campo mutex a cada nodo, esto llevara un mayor almacenamiento
    en memoria que podría ser considerable a la perfomance de la estructura.

    \subsection{Solución 3: Bloqueo de lectura y escritura}
    Para el caso de Member se realiza lectura, pero para Insert y Delete se realiza
    bloqueo por escritura. Para esto se crea bloqueos mas que todo en escritura y en la
    lectura se dejan pasar los threads ya que no hace modificaciones críticas a la
    estructura.

    \begin{lstlisting}[
        language = C++,
        caption = Solución 3 bloqueo Read-Write
    ]
    pthread_rwlock_rdlock(&rwlock);
    Member(value);
    pthread_rwlock_unlock(&rwlock);
    . . .
    pthread_rwlock_wrlock(&rwlock);
    Insert(value);
    pthread_rwlock_unlock(&rwlock);
    . . .
    pthread_rwlock_wrlock(&rwlock);
    Delete(value);
    pthread_rwlock_unlock(&rwlock);
    \end{lstlisting}

    \section{Tiempos}
    Para 8 threads:
    \begin{enumerate}
        \item Read-Write: -4.357145e-312 seconds
        \item Mute por lista: 1.484375e+00 seconds
        \item Mutex por nodo:
    \end{enumerate}

    \section{Hardware utilizado}
    \begin{enumerate}
        \item Intel Core i7
        \item 16 GB RAM
    \end{enumerate}
\end{document}